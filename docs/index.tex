% Created 2017-10-06 Fri 11:08
% Intended LaTeX compiler: pdflatex
\documentclass[presentation]{beamer}
\usepackage[utf8]{inputenc}
\usepackage[T1]{fontenc}
\usepackage{graphicx}
\usepackage{grffile}
\usepackage{longtable}
\usepackage{wrapfig}
\usepackage{rotating}
\usepackage[normalem]{ulem}
\usepackage{amsmath}
\usepackage{textcomp}
\usepackage{amssymb}
\usepackage{capt-of}
\usepackage{hyperref}
\usepackage{minted}
\usepackage[english]{babel}
\usetheme{default}
\author{Carl Henrik Ek}
\date{\today}
\title{}
\begin{document}

Welcome to the final year individual project page. Here you will find the information regarding the unit. This unit is different to other in that it is an \emph{individual} project, therefore the information here will be rather "blunt" as each and everyones journey will be different through the year. We will run a \href{http://github.com/carlhenrikek/COMSM0111/}{GitHub} repo where we will put things that are relevant for the unit. I will as much as possible also place links to these files directly from the webpage but keep in sync with the repo as very likely we'll forget to update the links.

The information on this page is for \alert{both} the third and the forth year project, some of the deadlines are different and some of the information are only related to one of the units.

\begin{frame}[label={sec:orga4a365d}]{Choosing a project}
The individual project is a real oppurtunity, you have the chance to spend a significant amount of time on something of your own choice. Choose well and it can be the most fun you've had during your degree, choose badly and you will have to spend significant time on something dull and worst of all, you can't even complain to others because it being individual you are the only one suffering in this specific way. \emph{Therefore choose wisely}.

The key thing is to really think through the idea before you start and that is what the first term is all about. Therefore try to think of what you are interested in, think of ideas, what units that you found to be fun and then try to come up with a couple ideas. Then the next step is to try and bounce the idea of other people. First talk to your friends, what do they think, try to shoot holes in the project. If you still feel that this is what you want to do then try to discuss the idea with academic staff people that you would like to be your supervisor.
\end{frame}

\begin{frame}[label={sec:org7f71b8f}]{Supervisor}
A supervisor is someone that oversees your work, someone to bounce ideas of and someone to get advice from with regards to methodology, writing etc. Importantly a supervisor is \alert{not} someone who directs your work, this is what you do. Your project will be a specialisation towards something and you will quickly know more about this specific thing than what your supervisor does.
\end{frame}

\begin{frame}[label={sec:org8fff392}]{Specification}
\alert{Draft version} You are supposed to hand in maximum one a4 page that describes in general terms the project that you intend to do. In specific you should address the points below, 

\begin{enumerate}
\item Title
\item Supervisor
\item Description of project
\item Evaluation or goal of project
\end{enumerate}

\alert{Final version (Deadline w12)} The final specification should be an extended version of the first and incorporate the feedback that you got from the first version. You should also make an outline of how you intend to approach the work and provide a brief schedule. In this version your supervisor should now be confirmed. The specification should not be more than two a4 pages.
\end{frame}

\begin{frame}[label={sec:orga9afbc9}]{Poster}
The department organises an annual project fair in March. This is an afternoon in which the final year students in both cohorts (BSc and MEng) will present their individual projects with posters and demos. A range of industrial partners will be in attendance. Undergraduates from other years are encouraged to attend too, particularly those in their penultimate year to get them thinking about what type of work they’d like to do in their final year.

All project students are expected to participate in the project fair. This involves preparing a poster, submitting it via SAFE \alert{one week} before the fair and discussing your work with interested parties on the afternoon. You may also want to think about preparing a demonstration of your project, e.g. on your laptop. Members of the marking panel will be in attendance and provide feedback on your work. As your project is still running, some of your results and achievements will be tentative or forthcoming: this is fine and expected.
\end{frame}
\begin{frame}[fragile,label={sec:org4274b4f}]{Dates}
 There are a couple of submissions during

\begin{center}
\begin{center}
\begin{tabular}{lrl}
Event & COMSM0111 & COMS30500\\
\hline
Introduce/welcome students to unit & 0 & 0\\
Submit initial specification & 6 & 5\\
Feedback initial specification & 7 & 7\\
Submit final specification & 12 & N/A\\
Confirm project and supervisor & 13 & N/A\\
Feedback on final specification & 13 & N/A\\
Lecture on spring work & 13 & N/A\\
Preliminary background chapter & N/A & 15\\
Poster session & March & March\\
Lecture on writing & before easter & N/A\\
Thesis submission & Mid-May & Mid-May\\
Thesis additional material & Mid-May & Mid-May\\
Oral presentation & TBD & TBD\\
\end{tabular}

\end{center}
\end{center}

\begin{minted}[]{python}
import 

\end{minted}
\end{frame}

\begin{frame}[label={sec:org0651d99}]{Lectures}
\begin{itemize}
\item COMSM0111 Welcome Lecture
\end{itemize}
\end{frame}


\begin{frame}[label={sec:orge57f171}]{COMSM0111 Group Project}
For the 4th year MEng project there exists the possibility of doing this as a group. These will be handled by Prof. David May so if you would be interested in this contact him directly.
\end{frame}
\end{document}
